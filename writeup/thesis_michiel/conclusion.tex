\chapter{Conclusion and Future Work\label{chap4}}

This system can successfully manage a complex set
of tasks with arbitrary dependencies. Tasks could be either be fully automated
by the system or could be completed by users. When user tasks are started, the
required files are incrementally transferred to the desktop host of the user
using \emph{rsync} to only transfer files that have not been transfered or are
out of date. To enforce quality and accuracy, a feature was added that enforces
that user tasks be validated by experienced members of the team before the tasks
can be labelled as complete.

Automated tasks are executed on the
server due to the fact that these tasks operate on very large files. By executing
them locally, data does not need to be transferred, which would be an expensive
process. Tasks are automatically started when all dependencies are met.

In the event that a task fails, the system also allows the user to inspect the
logging information that is generated during the execution of the task. Once the
problem is identified the tasks can then be manually restarted.

\section{Conclusion}
This research project was concluded with the successful implementation of a
Workflow Management System.  The design and
implementation was done in three iterations. This was explained in depth in
Chapter~\ref{chap2}.


The system was then successfully evaluated in Chapter~\ref{chap3} both for its
usability and its effectiveness at solving the problem. The following positive
results was obtained during the evaluation of the system:
\begin{enumerate}
    \item The system was successfully able to implement and execute a portion of
        the workflow in the modelling section of the modelling tasks that are
        present in the Zamani-Project. This sample workflow used a mix of system
        and user task.
    \item The system was positively evaluated using a sample group of 24 users.
        This evaluation revealed that users found the system useful, easy to
        use and users were satisfied using the system. User responses and the
        observations made during the test it was found that the system is
        effective and is very easy to learn.
\end{enumerate}

This system was however not implemented within the production Zamani Project. This was
mainly due to time constraints, caused by the scale and time required to
implement it. Functionally the system could be implemented, however this process
could be significantly simplified by the addition of some features. These are
mentioned in the future work session.


\section{Future Work}
During the implementation of the workflow system, various possible extensions
that could be added to the system could
not be implemented. These features would improve the system both in terms of
performance,
usability and set up time.
\begin{description}
\item[Hierarchical Workflows]\hfill \\
To allow better control and re usability over tasks, workflows should be
abstracted to include a hierarchy. Such a hierarchy would allow entire workflows
to be represented as singular nodes. These workflow, could then be repackaged
and reused in different sites, or even the same site. This would also allow the
setup for new sites to be much faster, as prepackaged workflows could easily be
used as drop-in components.
\item[Parameterized Scripts]\hfill \\
Oftentimes particular parameters of a script can change from one site to
another. This change does not necessarily affect the \emph{Task type}; however,
with the current implementation of the system the change would need to be made
at this point. This can be greatly improved by allowing a \emph{Task} to send
parameters to the job. This would require the Task Subsystem to allow parameters to
be sent dynamically to the \emph{Task Type}.
\item[Rule Based File Filters]\hfill \\
Currently within the system all the files in the output directory of are task is
treated as input to successor tasks. Tasks often only use a portion of the
files created by the predecessor. In order to currently facilitate this with the
system an additional filtering, task would need to be set up that filters out
unused files. By including a rule based filtering system much greater control
can be placed on the output files. Such rule based filters have been
successfully implemented in other systems\cite{conery2005rule}.
\item[Interactive Task Feedback Options]\hfill \\
In order to avoid one of the problems that were found in Section~\ref{eval:simple},
more interactivity is required for \emph{Tasks}. This primarily includes
real-time updates on the status of tasks. Further developments include the
ability to do more interactive validation such as discussion integration.
The addition of these collaborative tasks could resolves issues
with tasks in a uniform manner\cite{guimaraes1998integration}.
\item[Transformation-based Task Support]\hfill \\
Currently the system is built around creating derivative data items. However, it
is often common for certain files within a site to change, without creating an
additional copy. Although this behaviour is implicitly allowed, it should be
extended to be better defined within the system.
\item[Parallel Task Processing] \hfill \\
One of the most crucial aspects affecting the long term feasibility of the
system is its ability to scale and handle larger and more complicated workflows.
In this regard the server node would become a significant bottleneck in
processing \emph{Server Tasks}. In order to alleviate this problem, the system
would need to become distributed. This would present its own set of problems
as data would need to be efficiently distributed along the computation nodes
to ensure efficiency.

\end{description}

