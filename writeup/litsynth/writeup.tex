\documentclass[11pt,twocolumn]{article}
\usepackage[cm]{fullpage}


\title{Scientific Workbench and Workflow management for GIS workflow}

\author{
    Michiel Johan Baird \\
        Department of Computer Science \\
        University of Cape Town
}

\date{\today}

\begin{document}
\maketitle
\begin{abstract}
    This Literature synthesis does an overview of what has been done
    in the field of building a Scientific Workbench and Automated
    Workflow management. It focusses on implementation methods and
    case studies on where it has been implemented.

    This is then illustrates that this solution is highly applicable
    for GIS workflow.
\end{abstract}
\section{Introduction}
    Automated workflow management has been in wide use across
    various disciplines since the concept was formalised in in
    1996\cite{springerlink:10.1007/BF00136712}. Successful systems
    have been implemented across various fields including banking,
    pharmaceuticals and various others
    \cite{Brahe:2007:SWW:1316624.1316661,5407993}.

    This has been very successful in the field of science as
    the process can be rerun on different sets of data.\cite{4721191}
    This not only aids in reproducibility but also gives
    clear direction and saves time.

    WRITE SHORT PIECE ON GIS

\section{Overview}
    A workflow management system, also often reffered to as
    Grid Computing, consists of definitions on how a set of
    tasks should be executed\cite{springerlink:10.1007/BF00136712,vanderAalst2002125}.
    The overall procedure gets defined by the following
    components, actors, roles, responsibilities and
    obligations, tasks, activities, conseptual stuctures
    and resources.

    A real life problem or task can then be broken up to these
    components in such a way that the tasks represent a flow
    network. These tasks then connect to the actors and resources
    via the other components\cite[p.~4]{Taylor:2006:WES:1196459}.
    This aims to give a clear direction to a project.

    The initial system however almost immediately failed
    due to the fact that the system was far to rigid and
    more often than not change within the system was
    required.\cite{Suchman:1983:OPP:357442.357445}.

    These changes come from a number of sources including:
    ill-specification of initial problems, change in actors
    or resources, exceptions that occur and new requirements.
    Adaptive workflow systems were proposed to solve this
    problem by providing a mechanism for allowing change in
    the system. This allows processes to be extended,
    replaced or re-ordered. It also adds the ability change
    already running tasks by providing restart, transfer and
    proceed options\cite{vanderAalst2002125}.

    Scientific workflow management has also been very
    successful as it fits in very nice with how experiments
    are defined, and more importantly reused. Another
    benefit that was quickly discovered was that it also
    allowed researches to trade work flows making the
    replication of results much easier than they were
    previously. Keys to this success were that the workflow
    systems were made to fit the researchers, quick response
    to adding required features when needed, listening
    to user input and making sharing of workflows as easy as
    possible\cite{4721191}.

    Such a system has also been applied in fields that
    operate on large data sets, as would be the case if
    applied to GIS problems. Workflow systems were found
    to work well in the management of getting this data
    processed. Applying the consept to Observational
    astrophysics, it revealed that it could be used
    to identify bottle necks that could be optimised.
    Further it was used to automatically ensure local
    access of large files when it needed to be processed.
    \cite{Aragon:2009:WMH:1529282.1529491}

\section{Geographic Data}

\section{Implementations}

\section{Case Studies}

\section{Conclusion}


\bibliography{../references}{}
\bibliographystyle{acm}

\end{document}


