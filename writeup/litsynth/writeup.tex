\documentclass[12pt,twocolumn]{article}
\usepackage[cm]{fullpage}


\title{Scientific Workbench and Workflow management for GIS workflow}

\author{
    Michiel Johan Baird \\
        Department of Computer Science \\
        University of Cape Town
}

\date{\today}

\begin{document}
\maketitle
\begin{abstract}
    This Literature synthesis does an overview of what has been done
    in the field of building a Scientific Workbench and Automated
    Workflow management. It focusses on implementation methods and
    case studies on where it has been implemented.

    This is then illustrates that this solution is highly applicable
    for GIS workflow.
\end{abstract}
\section{Introduction}
    Automated workflow management has been in wide use across
    various disciplines since the concept was formalised in in
    1996\cite{springerlink:10.1007/BF00136712}. Successful systems
    have been implemented across various fields including banking,
    pharmaceuticals and various others
    \cite{Brahe:2007:SWW:1316624.1316661,5407993}.

    This has been very successful in the field of science as
    the process can be rerun on different sets of data.\cite{4721191}
    This not only aids in reproducibility but also gives
    clear direction and saves time.

    WRITE SHORT PIECE ON GIS

\section{Definition}

\section{GIS Data}

\section{Implementations}

\section{Case Studies}

\section{Conclusion}


\bibliography{../references}{}
\bibliographystyle{acm}

\end{document}


