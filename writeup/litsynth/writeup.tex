\documentclass[11pt,twocolumn]{article}
\usepackage[cm]{fullpage}
\usepackage{paralist}
\usepackage{color}
\usepackage{harvard}


\title{Scientific Workbench and Workflow management for GIS workflow}

\author{
    Michiel Johan Baird \\
        Department of Computer Science \\
        University of Cape Town
}

\date{\today}

\begin{document}
\maketitle
\begin{abstract}
    Automated workflow systems have been successfully implemented
    across various disciplines, including Scientific- and business
    workflows. This is an overview of what has has been done
    in the field. Special attention will be given
    to the building of a Scientific Workbench. It focusses previous efforts, highlighting some the methods used
    as well as the lessons learnt during the implementations.

    It also looks at how these principles could be applied
    specifically to GIS workflow by giving an overview of the
    structure of the field. It seeks to find an appropriate
    mapping to these systems using known principles of SOA.

    It is then illustrated that this solution is highly applicable
    to GIS workflow, provided the necessary middle-ware can be built
    to facilitate integration.
\end{abstract}
\section{Introduction}
    Workflow management systems define a complex process in
    into well formed tasks and coordinates the process
    completion \cite{1245778}.
    Automated workflow management has been in wide use across
    various disciplines since the concept was formalised in
    1996\cite{springerlink:10.1007/BF00136712}. Successful systems
    have been implemented across various fields including banking,
    pharmaceuticals and various others
    \cite{Brahe:2007:SWW:1316624.1316661,5407993}.

    It has been shown to be very successful in the field
    of science as the same scientific process can easily
    be repeated on
    a different set of data.\cite{4721191}
    This not only aids in reproducibility but also gives
    clear direction and saves time. This is done by efficiently
    abstracting the operations in the flow, allowing it
    to be automatically handled.

    Geographic Information Systems(GIS) is the field that
    concerns itself with the organisation, representation
    and processing of geographic data, for the purpose of
    querying it and making decissions off of the data
    \cite{DiMartino:2007:TAG:1341012.1341081}. The
    workflow in GIS is very distributed and the set
    of data that is operated on is large and diverse.
    Workflow management within GIS has been considered
    and solutions have been proposed but has not implemented
    or evaluated\cite{Migliorini:2011:WTG:1999320.1999356}.


\section{Overview}
    A workflow management system, consists of definitions
    on how a set of tasks should be executed
    \cite{springerlink:10.1007/BF00136712,vanderAalst2002125}.
    The overall procedure is defined by the following
components: \begin{inparaenum}[(i)] \item actors,
    \item roles, \item responsibilities and
    obligations, \item tasks, \item activities,\item conceptual structures
    and \item resources.\end{inparaenum}

    A real life problem or task can then be broken up into these
    components in such a way that the tasks represent a flow
    network. These tasks then connect to the actors and resources
    via the other components\cite[p.~4]{Taylor:2006:WES:1196459}.
    This aims to give a clear direction to a project, and
    allows task to be executed efficiently in a distributed manor.

    The initial implementations of a workflow system
    however, almost immediately failed. The system was
    too rigid an was unable to accommodate the high levels
    of change that was required by the users.
    \cite{Suchman:1983:OPP:357442.357445}.

    These changes come from a number of sources, including:
    ill-specification of initial problems, change in actors
    or resources, exceptions that occurred and new requirements.
    Adaptive workflow systems were proposed to solve this
    problem by providing a mechanism for allowing change in
    the system. This allows processes to be extended,
    replaced or re-ordered. It also adds the ability to change
    already running tasks by providing restart, transfer and
    proceed options\cite{vanderAalst2002125}.

    Scientific workflow management has also been very
    successful with how experiments
    are defined, and more importantly reused. Another
    benefit that was quickly discovered was that it also
    allowed researchers to trade work flows making the
    replication of results much easier than they were
    previously. Keys to this success were: that the workflow
    systems were made to fit the researchers, quick responses
    to adding required features when needed, listening
    to user input and making sharing of workflows as easy as
    possible\cite{4721191}.

    Such a system has also been applied in fields that
    operate on large data sets, as would be the case if
    applied to GIS problems. Workflow systems were found
    to work well in the management of getting this data
    processed. Applying the concept to Observational
    astrophysics, it revealed that it could be used
    to identify bottle necks that could be optimised.
    Further it was used to automatically ensure local
    access of large files that needed to be processed.
    \cite{Aragon:2009:WMH:1529282.1529491}

\section{Geographic Data}
    GIS concerns itself with the collection, organisation
    and query of geographic data. \cite{DiMartino:2007:TAG:1341012.1341081}
    This data includes but  is not limited to landscapes,
    coordinate data, building models,
    statistics, pictures, textures and routes. This
    is a very broad set of data, varying from very large to very small.
    That variation however means that there exists no uniform method
    to efficiently deal with the data.

    The processing of this data can vary from human, to software
    processing. Various Web applications have been written
    to facilitate the tasks that need to be accomplished.
    This software is known as WebGIS and is becoming more
    popular with scientists; it also means that even within
    the field there is a strong shift toward Web based services
    \cite{DiMartino:2007:TAG:1341012.1341081}.

    A key realisation with the usage of this data is that
    the same data is used across various applications,
    to create various amounts of  abstractions.
    The core data is seldom changed. Instead a new abstraction
    layer is added on top of it. This allows that the data
    can be thought of as a graph, where the nodes represent
    either a data- or abstraction element, and the edges
    represents the functions/tasks required to create the
    particular abstraction as a set of topological
    relationships. This can be effectively used to provide
    high levels of GIS
    interoperability\cite{ElAdnani:2001:MLF:512161.512177}.

\section{Implementations}
    There are various products available that can compose
    scientific workflows. \emph{The Trident workbench}
    \cite{Simmhan:2009:BTS:1673063.1673121} is an open
    source workflow management system developed by Microsoft
    Research that also adds middleware services and a graphical
    composition interface. Trident builds work flows of control
    and data flows, off of built-in, user defined activities and
    nested subflows.

    The flows are represented using XOML (XML Specification) while
    the activities are stored as a set of sub routines. Trident
    can be used on a local system, remote systems and even clusters.
    Queries on the system can be performed using LINQ.
    \cite{Simmhan2011790}

    \emph{Kepler} is another scientific workflow
    management system that provides workflow design and execution.
    Actors are designed to perform independent tasks that can either
    be atomic or  composite. Composite actors(subflows) consist
    of multiple   atomic actors bundled together. Actors can consume data and
    produce output, called tokens. Actors communicate tokens with
    each other via links. The order of execution and the links are
    defined by an independent entity called the director. As a
    consequence, the workflow can either be executed in a
    sequential or parallel manner. Kepler effectively separates
    the workflow from its execution, allowing for easy batch
    execution. Actors can easily be exported and shared.
    Kepler is very popular due to its adaptability and easy
    integration. \cite{Wang:2009:KHG:1645164.1645176}

    \emph{Taverna} is another scientific workbench that supports
    application-level workflow and does not focus on scheduling
    as much others. Taverna has a strong focus on workflow
    sharing. Taverna is quite popular, since there exists
    a social network, designed to facilitate workflow sharing
    between scientists(\emph{myExperiment}). Services are linked to the model to
    execute the various tasks. Taverna can be used in such
    a way that it can utilize all the services a client has
    to facilitate the flow by easily adding services. The
    taverna language is a simple data-flow language called
    the Simple Conceptual Unified Language(SCUFL) that can
    be encoded to XML\cite{4721191}.

    In order for these workbenches to be successful there needs
    to exists a high level of interoperability between the
    workflow management and the services that are required.
    However, due to the fact that there is a relatively high
    chance of failure when building this interoperability into the
    services as a core component, is an extremely high risk
    and therefore is not typically done. Cheaper ways of
    doing this is providing middleware that can wrap around
    the service to provide the required interfaces
    \cite{Shegalov:2001:XWM:767132.767139}.
    This need for interoperability has led to the
    popularisation SOA(Service Orientated Architecture).
    Although the concept has been around since the 1970s,
    it has only recently gained favour due to Web services.
    Web services are software that run on the internet through
    XML standards-based interfaces. Each service provides
    a fuctional description using the \emph{Web Services
    Description Language}(WSDL). This description provides
    the supported operations, as well as the definition
    of the input and output messages.\cite{Tai:2004:CCW:1045658.1045680}
    It should be noted that SOA is \emph{not} an implementation,
    but rather an \emph{Architectural Model}
    SOA refers to a collection of loosely coupled services,
    that individually carry out a particular process. Each
    service should have a well defined interface with self-contained
    functionality. It should allow other applications
    or services to use this functionality without knowing
    the underlying technical details. These services should be
    hidden from the end-user and its usage should preferably
    be platform independent.
    \cite{Sanders:2008:SSA:1400549.1400595}.

    By using the concepts from SOA, a workflow system can
    be built that automatically uses these Web Services
    to facilitate both the data and control flow using
    well defined interfaces in standards such as XML/JSON.
    \cite{Shegalov:2001:XWM:767132.767139}. With the
    advancement of WebGIS, a lot  of Web Services already
    exists that facilitates GIS processing.


\section{Case Studies}
    The next section will look at two instances where
    workflow management systems were implemented and used.
    These case studies will look at both a business and a
    scientific application.
    \subsection*{Danske Bank}
      The workflow management system at \emph{Danske bank} was
      incrementally implemented as their system moved
      from a manual system.\cite{Brahe:2007:SWW:1316624.1316661}

      This system was developed as an in house solution when
      the manual system could not cope anymore.
      Several lessons were learned during the process that
      is applicable to other work flow systems. When work
      was divided purely from an efficiency point of view
      the workers became complacent as they felt that they
      did not understand the overall mechanism and felt that
      they were not involved. They discovered that the
      system did not handle change very well. This change
      was expensive and inevitable. Their system had
      to be adapted to handle this change. The success of the
      system is mainly attributed to the interoperability and
      close relationship between the users and the developers

    \subsection*{OrthoSearch}
      \emph{OrophoSearch} is a workflow,
      built on \emph{Kepler} that is designed to work on
      work on data in the field of Bio Informatics.
      \cite{daCruz:2008:OSW:1363686.1363983}

      A workflow system was implemented in \emph{Kepler}
      as it addressed the requirements they had
      including: \begin{inparaenum}[(i)]
      \item Workflow definition and Design; \item workflow execution
      control; \item fault tolerance; \item intermediate
      data management; and \item data provenance support.
      \end{inparaenum}

      Although the system was not without its hiccups and changes;
      the integration with Kepler provided the workflow with the
      direction and increased overall productivity.


\section{Conclusion}
   The field of GIS concerns itself with a vast amount of Geographic
   data. This data comes in various sizes and as such different
   methods of handling and transferring would need to be used to
   facilitate dataflows within the system. It was also found that
   there are a large number of transformations that workflow would
   need to support.

   The work however is done in a very distributed manner, which allows
   for a very effective mapping onto a grid-based computing solution,
   provided middleware can be developed to support the systems that
   are used. This would allow for an  effective Content Delivery Network
   that provides data on demand where it is needed on the grid
   \cite{Montella:2007:UGC:1272980.1272995}.

   GIS workflow, due to its distributed nature, would map
   well onto a automated workflow system. The nature of the science
   is supported well. It would allow for effective automatisation
   of some of the functions are available. The incorporation of flows
   would also be effectively deal with
   the
   problems associated with Content Delivery.
   \cite{Withana:2010:VWE:1851476.1851586}

\bibliography{../references}{}
\bibliographystyle{agsm}

\end{document}


