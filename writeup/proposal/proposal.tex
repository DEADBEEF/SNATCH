\documentclass[12pt,a4paper]{article}
%\usepackage{geometry}[paper=a4paper,left=30mm,width=150mm,top=25mm,bottom=25mm]
\usepackage[margin=2cm]{geometry}
\usepackage[utf8]{inputenc}
\usepackage{amsmath}
\usepackage{amsfonts}
\usepackage{amssymb}
\begin{document}
\author{Timothy Trewartha \\
\and
Michiel Johan Baird \\
\and
Supervisor: Hussein Suleman \\
Department of Computer Science \\
University of Cape Town
 }
\title{Spatial Navigation of Cultural Heritage Sites}
\maketitle
\abstract{Do we need an abstract?}
\tableofcontents
\newpage

\section{Introduction}
\section{Project Description}
\section{Problem Statement}
\subsection{Architectural Model Streaming}
\subsection{Workflow and Data Management}
Geographic Information Science involves the capture, storage, manipulation and analysis
and management of geographic data. This data is very diverse and as such has to be handled
in quite diverse ways. This data gets abstracted into various forms. This presents a
rather unique challenge in managing the data as it could be used by anyone of the research
staff at any point in the process. This data movement is laborious and could benefit from
from automisation.

Workflow Management Systems aim to decompose compilated projects and processes into
small atomic chucks. This decompisition can then be optimised to improve the efficiency.
GIS research projects are generally done with multiperson teams where the work is
done in a parallel fashion. Under these conditions workflow management systems
are optimal.

The aim is to provide a workflow management that is applicable for GIS projects.
This system should be able to: interface with the current systems; track and
manage the workflow; provide local data availabilty and content delivery; and
increases overall efficiency within the dicipline.

\section{Procedures and Methods}
\section{Ethical, Professional and Legal Issues}
At this point we see no legal/ethical issues
\section{Related Work}
\section{Anticipated Outcomes}
\section{Project Plan}
\subsection{Risks}
\subsubsection*{Network Constraints}
\subsubsection*{Middleware}
For this project to be successful, the WFMS it would have to interface
heavily with existing software used to perform
\subsubsection*{Hardware Limitations}
\subsubsection*{Large Indices}
\subsubsection*{Implementation Difficulties}
\subsection{Timeline, including Gantt chart}
We need a nice program for creating Gantt charts
\subsection{Resources required}
\subsubsection*{Hardware}
\subsubsection*{Geographic Data}
\subsection{Deliverables}
\subsubsection{GIS Workbench}
\subsubsection{Data Flow Facilitator}
\subsubsection{Integration with WFMS}
Workflow Management System
\subsubsection{Hierarchical Data Structure}
\subsubsection{Level of Detail Streaming}
\subsection{Milestones}
\subsubsection{Architectural Model Streaming}
\subsubsection*{Datastructure Implemented}
\subsubsection{Workflow and Data Management}
\subsubsection{Combined}
\subsection{Work Allocation}
Project is well divided already, we know who's doing what.
\section{Conclusion}




\end{document}
