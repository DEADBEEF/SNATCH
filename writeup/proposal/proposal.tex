\documentclass[12pt,a4paper]{article}
%\usepackage{geometry}[paper=a4paper,left=30mm,width=150mm,top=25mm,bottom=25mm]
\usepackage[margin=2cm]{geometry}
\usepackage[utf8]{inputenc}
\usepackage{amsmath}
\usepackage{amsfonts}
\usepackage{amssymb}
\begin{document}
\author{Timothy Trewartha \\
\and
Michiel Johan Baird \\
\and
Supervisor: Hussein Suleman \\
Department of Computer Science \\
University of Cape Town
 }
\title{Spatial Navigation of Cultural Heritage Sites}
\maketitle
\abstract{Do we need an abstract?}
\tableofcontents
\newpage

\section{Project Description}
The Zamani project, started by the UCT Department of Geomatics, aims to preserve African cultural heritage by documenting heritage sites and producing laser scanned models.
\section{Problem Statement}
\subsection{Architectural Model Streaming}
The UCT Department of Geomatics has indicated that they have difficulties handling the size of some of their models. These laser scanned models of cultural heritage sites are often very large, some of them containing over 8 billion points. Given this vast scale of data, traditional viewing methods and the current hardware and software systems are not able to cope.  Consequently, before viewing or manipulating the data, one must go through a process of decimating the original data by a factor of 10, 100 or more. This compromise is often unacceptable, as one may often require the full original detail. This level of detail is often necessary for cultural heritage sites in order to view details such as cracks and flaws, with a view to preserving the site and preventing damage.
\subsection{Workflow and Data Management}
Geographic Information Science involves the capture, storage, manipulation and analysis
and management of geographic data. This data is very diverse and as such has to be handled
in quite diverse ways. This data gets abstracted into various forms. This presents a
rather unique challenge in managing the data as it could be used by anyone of the research
staff at any point in the process. This data movement is laborious and could benefit from
from automisation.

Workflow Management Systems aim to decompose compilated projects and processes into
small atomic chucks. This decompisition can then be optimised to improve the efficiency.
GIS research projects are generally done with multiperson teams where the work is
done in a parallel fashion. Under these conditions workflow management systems
are optimal.

The aim is to provide a workflow management that is applicable for GIS projects.
This system should be able to: interface with the current systems; track and
manage the workflow; provide local data availabilty and content delivery; and
increases overall efficiency within the dicipline.

\section{Procedures and Methods}
\subsection{Hierarchical Data Structure}

\section{Ethical, Professional and Legal Issues}
At this point we see no legal/ethical issues
\section{Related Work}
\section{Anticipated Outcomes}
\section{Project Plan}
\subsection{Risks}
\subsubsection*{Network Constraints}
One of the core functionalities of the system would be to provide
content delivery of data that is required for a specific task. Providing
this local data allows the task to get completed without unnessesry
fetching delays. There is a risk that this content delivery system
could saturate the network. This would cause the system to be slow
and unusable.
\subsubsection*{Middleware}
For this project to be successful, the WFMS it would have to interface
heavily with existing software used to perform GIS operations. This will
require large amounts of middleware to be developed that understand the
the input and output formats of this software. Since many of these
formats are propriatary a significant amount of effort will have to
be made for the sytem to function. If these formats can not be intergrated
it presents a huge risk to the project.
\subsubsection*{Hardware Limitations}
There is a risk that the hardware available will not be able to cope
with the load that will be required. Since a distributed system is
not being proposed there is a risk that the system will become a bottleneck.
\subsubsection*{Large Indices}
When indexing the models we will have to generate a significant amount of data. Given that many of the models are already very large, these indices might become unfeasibly large. Dealing with such large indexes will be an important part of the project and we will have to deal with this risk.
\subsubsection*{Integration with Existing GIS Software}
The hierarchical datastructure required as part of this project aims to facilitate level of detail streaming and realtime interaction. However, ideally one should not have to re-implement tools which are already available such as ArcGIS. We aim to integrate our datastructure into a pre-existing software package to prevent unnecessary work. However, this may be difficult and there are several associated risks such as, unavailability of source code, lack of documenation for the software, and potential copyright license infringement.
\subsubsection*{Implementation Difficulties}
\subsection{Timeline, including Gantt chart}
We need a nice program for creating Gantt charts
\subsection{Resources required}
\subsubsection*{Hardware}
\subsubsection*{Geographic Data}
\subsection{Deliverables}
\subsubsection{GIS Workbench}
\subsubsection{Data Flow Facilitator}
\subsubsection{Integration with WFMS}
Workflow Management System
\subsubsection{Hierarchical Data Structure}
\subsubsection{Level of Detail Streaming}
\subsection{Milestones}
\subsubsection{Architectural Model Streaming}
\subsubsection*{Datastructure Implemented}
\subsubsection*{System is able to render large models}
\subsubsection*{System is able to stream large models from the server}
\subsubsection{Workflow and Data Management}
\subsubsection{Combined}
\subsection{Work Allocation}
Project is well divided already, we know who's doing what.
\section{Conclusion}




\end{document}
